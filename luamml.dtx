% \iffalse meta-comment
%
%% Copyright (C) 2020-2021 by Marcel Krueger
%%
%% This file may be distributed and/or modified under the
%% conditions of the LaTeX Project Public License, either
%% version 1.3c of this license or (at your option) any later
%% version. The latest version of this license is in:
%%
%% http://www.latex-project.org/lppl.txt
%%
%% and version 1.3 or later is part of all distributions of
%% LaTeX version 2005/12/01 or later.
%
%<*batch>
%<*gobble>
\ifx\jobname\relax\let\documentclass\undefined\fi
\ifx\documentclass\undefined
\csname fi\endcsname
%</gobble>
\input docstrip.tex
\keepsilent
\generate{\file{luamml.sty}{\from{luamml.dtx}{package}}}
\endbatchfile
%</batch>
%<*gobble>
\fi
\expandafter\ifx\csname @currname\endcsname\empty
\csname fi\endcsname
%</gobble>
%<*driver>
\documentclass{l3doc}
\usepackage{luamml}
\usepackage{csquotes,luacolor}
\RecordChanges
\begin{document}
\tracingmathml2
\DocInput{luamml.dtx}
\PrintIndex
\PrintChanges
\end{document}
%</driver>
%<*gobble>
\fi
%</gobble>
% \fi
%
% \title{The \pkg{luamml} package}
%
% \author{Marcel Krüger}
%
% \maketitle
%
% \begin{documentation}
% \end{documentation}
%
% \begin{implementation}
% \section{Package Implementation}
% \iffalse
%<*package>
% \fi
%    \begin{macrocode}
%<@@=luamml>
\ProvidesExplPackage {luamml} {2021-04-23} {0.0.1-alpha}
  {Automatically generate presentational MathML from LuaTeX math expressions}
%    \end{macrocode}
%
% \subsection{Initialization}
% These variable have to appear before the Lua module is loaded and will be used to
% communicate information to the callback.
%
% Here \cs{tracingmathml} does not use a expl3 name since it is not intended for
% programming use but only as a debugging helper for the user.
% The other variables are internal, but we provide public interfaces for setting
% them later.
%    \begin{macrocode}
\int_new:N \l__luamml_flag_int
\tl_new:N \l__luamml_filename_tl
\tl_new:N \l__luamml_root_tl
\tl_set:Nn \l__luamml_root_tl { mrow }
\int_new:N \tracingmathml
%    \end{macrocode}
%
% Now we can load the Lua module which defines the callback.
%    \begin{macrocode}
\lua_now:n { require'luamml-tex' }
%    \end{macrocode}
%
% \subsection{Flags}
% The most important interface is for setting the flag which controls how the
% formulas should be converted.
%
% \begin{macro}{\luamml_flag_process:}
%   Consider the current formula to be a complete, free-standing mathematical
%   expression which should be converted to MathML. Additionally, the formula
%   is also saved in the \texttt{start\_math} node as with
%   \cs{luamml_flag_save:}.
%    \begin{macrocode}
\cs_new_protected:Npn \luamml_flag_process: {
  \int_set:Nn \l__luamml_flag_int { 3 }
}
%    \end{macrocode}
% \end{macro}
%
% \begin{macro}{\luamml_flag_save:,
%               \luamml_flag_save:N,
%               \luamml_flag_save:n,
%               \luamml_flag_save:Nn}
%   Convert the current formula but only save it's representation in the math
%   node without emitting it as a complete formula. This is useful when the
%   expression forms part of a bigger formula and will be intergrated into it's
%   MathML tables later by special code.
%   It optinally accepts two parameters: One math style command
%   (\cs{displaystyle}, \cs{textstyle}, etc.) which is the implicit math style
%   (so the style which the surrounding code expects this style to have) and a
%   name for the root element (defaults to \texttt{mrow}).
%   If the root element name is \texttt{mrow}, it will get suppressed in some
%   cases.
%    \begin{macrocode}
\cs_new_protected:Npn \luamml_flag_save: {
  \int_set:Nn \l__luamml_flag_int { 1 }
}
\cs_new_protected:Npn \luamml_flag_save:N #1 {
  \int_set:Nn \l__luamml_flag_int { 17 + 32 * #1 }
}
\cs_new_protected:Npn \luamml_flag_save:n {
  \int_set:Nn \l__luamml_flag_int { 5 }
  \tl_set:Nn \l__luamml_root_tl
}
\cs_new_protected:Npn \luamml_flag_save:Nn #1 {
  \int_set:Nn \l__luamml_flag_int { 21 + 32 * #1 }
  \tl_set:Nn \l__luamml_root_tl
}
%    \end{macrocode}
% \end{macro}
%
% \begin{macro}{\luamml_flag_ignore:}
%   Completely ignore the math mode material.
%    \begin{macrocode}
\cs_new_protected:Npn \luamml_flag_ignore: {
  \int_set:Nn \l__luamml_flag_int { 0 }
}
%    \end{macrocode}
% \end{macro}
%
% \begin{macro}{\luamml_flag_structelem:}
%   Like \cs{luamml_flag_process:}, but additionally add PDF structure
%   elements. This only works if \pkg{tagpdf} has been loaded \emph{before}
%   \texttt{luamml}.
%    \begin{macrocode}
\cs_new_protected:Npn \luamml_flag_structelem: {
  \int_set:Nn \l__luamml_flag_int { 11 }
}
%    \end{macrocode}
% \end{macro}
%
% \begin{macro}{\luamml_set_filename:n}
%   Allows to set a filename to which the generated MathML gets written.
%   Previous content from the file will get overwritten. This includes results
%   written by a previous formula. Therefore this has to be called separately
%   for every formula or it must expand to different values to be useful.
%   The value is fully expanded when the file is written.
%   
%   Only complete formulas get written into files (so formulas where
%   \cs{luamml_flag_process:} or \cs{luamml_flag_structelem:} are in effect).
%    \begin{macrocode}
\cs_new_protected:Npn \luamml_set_filename:n {
  \tl_set:Nn \l__luamml_filename_tl
}
%    \end{macrocode}
% \end{macro}
%
% By default, the flag is set to assume complete formulas.
%    \begin{macrocode}
\luamml_flag_process:
%    \end{macrocode}
%
% \subsection{Annotations}
% \begin{macro}{\luamml_annotate:nen, \luamml_annotate:en}
% A simple annotation scheme: The first argument is the number of top level
% noads to be annotated, the second parameter the annotation and the third
% parameter the actual list of math tokens. The first argument can be omitted to
% let Lua\TeX detrmine the number itself.
%
% Passing the first parameter explicitly is useful for any annotations which
% should be compatible with fututre pdf\TeX versions of this functionality.
%    \begin{macrocode}
\cs_new_protected:Npn \luamml_annotate:nen #1#2#3 {
  \__luamml_annotate_begin:
    #3
    \__luamml_annotate_end:we \tex_numexpr:D #1 \scan_stop: {#2}
}

\cs_new_protected:Npn \luamml_annotate:en #1#2 {
  \__luamml_annotate_begin:
    #2
  \__luamml_annotate_end:e {#1}
}
%    \end{macrocode}
% \end{macro}
%
% \subsection{Patching}
% For some packages, we ship with patches to make them more compatible and to
% demonstrate how other code can be patched to work with \texttt{luamml}.
% 
% These are either loaded directly if the packages are loaded or delayed using
% \LaTeX's hook system otherwise.
% \begin{macro}{\__luamml_patch_package:nn, \__luamml_patch_package:n}
% For this, we use two helpers: First a wrapper which runs arbitrary code either
% now (if the package is already loaded) or as soon as the package loads, second
% an application of the first one to load packages following \texttt{luamml}'s
% naming scheme for these patch packages.
%    \begin{macrocode}
\cs_new_protected:Npn \__luamml_patch_package:nn #1 #2 {
  \@ifpackageloaded {#1} {#2} {
    \hook_gput_code:nnn {package/after/#1} {luamml} {#2}
  }
}
\cs_new_protected:Npn \__luamml_patch_package:n #1 {
  \__luamml_patch_package:nn {#1} {
    \RequirePackage { luamml-patches-#1 }
  }
}
%    \end{macrocode}
% \end{macro}
%
%    \begin{macrocode}
\RequirePackage { luamml-patches-kernel }
\__luamml_patch_package:n {amsmath}
\__luamml_patch_package:n {array}
%    \end{macrocode}

% \iffalse
%</package>
% \fi
% \end{implementation}
% \Finale
